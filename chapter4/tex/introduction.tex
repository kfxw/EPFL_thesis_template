
\section{Introduction}
Geometry parameterization and manipulation are central to aerodynamic shape optimization problem.
However, effective and efficient parameterization often requires significant human intervention, hindering full automation of shape optimization.
It happens for several reasons.
First, mesh parameterization and hyperparameters depend heavily on the object of interest, requiring geometric prior knowledge and case-by-case analysis.
For example, determining the number and position of control points for the Free Form Deformation (FFD) \cite{aa.Sederberg1986, aa.Lamousin1994, aa.Kenway2010} and Nonuniform Rational B-Splines (NURBS) \cite{aa.Toal2010} methods, and the number of polynomial bases for Class/Shape function Transformation (CST)~\cite{aa.Kulfan2008} is critical to the performance.
Second, conventional method only parameterize object surfaces, which leads to detachment of the computational mesh from the original surface, necessitating time-consuming remeshing or additional mesh deformation algorithms \cite{aa.deBoer2007, aa.Batina1990, aa.Batina1991, aa.Farhat1998, aa.Luke2012} that increase computational time and require additional tuning of hyperparameters.
Third, conventional methods are less capable in handling the high dimensionality. 
Dimension reduction methods are often applied to mitigate the curse of dimensionality \cite{ai.Bellman1961} but they also introduce new problems.
For example, the use of Proper Orthogonal Decomposition (POD) methods \cite{aa.Robinson2001,aa.Poole2015, aa.Masters2017,aa.Li2019,aa.Kedward2020} are usually coupled with task-specific handcraft engineering to reconstruct the geometry from a latent vector.
The Active Subspace methods \cite{aa.Constantine2014,aa.Li2019b,aa.Lukaczyk2014,aa.Namura2017,aa.Grey2018,aa.Bauerheim2016,aa.Magri2016} exploit the surrogate models' gradients, but the upper bound of its approximation error increases with the dimensionality.
The Generative Topographic Mapping (GTM) method~\cite{aa.Viswanath2011} faces an exponential increase in the number of the radius basis to tune manually.
Fourth, none of the existing dimension reduction approaches to automate parameterization \cite{aa.Robinson2001, aa.Poole2015, aa.Masters2017, aa.Li2019, aa.Kedward2020, aa.Viswanath2011, aa.Constantine2014} are designed to be differentiable, making seamless integration into gradient-based frameworks, such as those that involve deep learning or adjoint CFD solvers, impossible.

More recently, several works have emphasized the significance of using deep learning techniques in shape parameterization \cite{aa.Li2020, aa.Li2021,aa.Li2022d} and dimension reduction \cite{aa.Glaws2022}. However, they are subject to similar limitations as conventional methods, including the need for geometric priors in model design or data processing, non-differentiability with respect to design variables, and the reliance on conventional parameterization methods.

The main contribution of our work is a novel approach to shape parameterization that (i) eliminates the need for handcrafted parameters, (ii) incorporates computational mesh deformation to parameterize the entire CFD mesh, (iii) can directly handle high-dimensional mesh data for maximizing the freedom of geometry manipulation, and (iv) is fully differentiable, enabling straightforward implementation in gradient-based frameworks used for shape optimization or uncertainty quantification studies. We firstly drew inspiration from computer vision research, which has demonstrated the effectiveness of implicit representations \cite{ai.Park2019c} that eliminate explicit shape parameterization, and develop models for continuous mesh representation. We propose and investigate two models: the Latent Space Model (LSM) and the Direct Mapping Model (DMM). Both models encode the geometric information of the object of interest as a deformation from a fixed shape template and decode the parameterized design variables into an entire CFD mesh. The two models employ different training strategies and utilize distinct amounts of training data, resulting in unique properties and applicability to different tasks. Furthermore, we devise a novel regularization loss function that acts as an implicit optimizer, guiding both LSM and DMM to properly deform the CFD mesh to preserve the mesh quality for downstream numerical simulation.

The remainder of this paper is structured as follows. Section \ref{ch4:sec:method} will provide a detailed description of both models, including their structures, training and inference. Specifically, Section 1.3 will establish the formulation of the novel regularization loss function, and Section \ref{ch4:sec:discussion_lsm_dmm} will discuss the applicable scenario of LSM and DMM. In Section \ref{ch4:sec:final_exp_res}, we demonstrate the use of LSM and DMM coupled with a differentiable surrogate model and an adjoint solver for 2D airfoil shape optimization.
