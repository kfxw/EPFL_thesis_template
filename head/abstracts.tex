%\begingroup
%\let\cleardoublepage\clearpage


% English abstract
\cleardoublepage
\chapter*{Abstract}
\markboth{Abstract}{Abstract}
\addcontentsline{toc}{chapter}{Abstract (English/Français)} % adds an entry to the table of contents

Engineering design is the process through which people draft, evaluate and optimize products, and it forms the foundation of manufacturing. Modern engineering design increasingly demands automated, scalable methods that handle complex geometries and high-dimensional design spaces while accelerating the entire process. This thesis introduces a series of deep geometric learning techniques that provide flexible, differentiable shape representations and enable controllable design exploration, focusing on aerodynamic shape design and optimization. 

First, I develop a Latent Space Model (LSM) for airfoil Computational Fluid Dynamics (CFD) meshes, an auto-decoder network that encodes each mesh into a low-dimensional latent code and reconstructs it by deforming a fixed template. A novel regularization ensures smooth, valid deformed CFD meshes. Because the mesh geometry remains fully differentiable with respect to the latent code, the LSM allows efficient gradient-based optimization directly through the mesh. Next, I introduce and compare two parameterization strategies. The LSM learns deformation from a database of shapes, while the Direct Mapping Model (DMM) constructs a parameterization on the fly for a single target geometry without training data. Both models incorporate mesh-regularized deformation to preserve computational mesh quality. The analysis clarifies their trade-offs: LSM leverages data priors whereas DMM is data-independent, and each can be chosen based on specific needs.

Building on these insights, I develop DeepGeo that generalizes DMM to complex 3D geometries by automatically learning deformation in high-dimensional spaces, thereby streamlining the shape parameterization in design optimization. It provides large deformation freedom while inherently enforcing global smoothness. In case studies, including a 2D circle, the NASA Common Research Model wing, and a Blended-Wing-Body aircraft, DeepGeo produces optimized shapes with aerodynamic performance on par with state-of-the-art hand-crafted parameterizations, while significantly reducing manual effort.

The thesis then shifts to design space exploration, a necessary step before design optimization for concept prototyping. I propose DiffGeo, a latent-space diffusion model that generates diverse, valid aerodynamic shapes. By training the diffusion model on the latent codes learned with LSM, DiffGeo achieves high data efficiency and guarantees validity of generated designs. DiffGeo also demonstrates conditional sampling: by guiding the diffusion process with complex geometric objectives, engineers can generate shapes meeting specified criteria. Finally, I participated in developing Dflow-SUR, a physics-guided diffusion approach with high computational efficiency and superior controllability. Optimizing the sampled noises through gradient feedback from a surrogate model produces airfoils and wings with improved aerodynamic performance.

These contributions together form a coherent design pipeline: from efficient design space exploration to highly automated design optimization empowered by fully differentiable parameterization models. The results show that deep geometric learning can handle challenging design representations and greatly accelerate the design cycle by automating much of the geometry handling and providing rapid concept generation. This interdisciplinary research paves the way toward more exploratory, efficient and human-AI collaborative design workflows.

\textbf{Key words:} AI-driven engineering design, deep geometric learning, aerodynamic shape optimization, shape parameterization, generative design

% French abstract
\begin{otherlanguage}{french}
\cleardoublepage
\chapter*{Résumé}
\markboth{Résumé}{Résumé}

La conception ingénierique est le processus par lequel on élabore, évalue et optimise des produits, et elle fonde la fabrication moderne. La conception moderne exige des méthodes automatisées et évolutives capables de traiter des géométries complexes et des espaces de conception de haute dimension tout en accélérant l’ensemble du processus. Cette thèse présente une série de techniques d’apprentissage profond géométrique offrant des représentations de forme flexibles et différentiables et une exploration contrôlable de l’espace de conception, axée sur la conception et l’optimisation aérodynamiques.

Dans un premier temps, je développe d’abord un Latent Space Model (LSM) pour les maillages CFD de profils aérodynamiques: un réseau auto-décodeur qui encode chaque maillage dans un code latent de faible dimension et le reconstruit en déformant un gabarit fixe. Une régularisation originale garantit des maillages CFD lisses et valides. La géométrie du maillage restant entièrement différentiable par rapport au code latent, le LSM permet une optimisation efficace par gradients directement à travers le maillage. J’introduis ensuite deux stratégies de paramétrisation et les compare: le LSM apprend la déformation à partir d’une base de formes, tandis que le Direct Mapping Model (DMM) construit une paramétrisation à la volée pour une géométrie cible, sans données d’entraînement. Les deux modèles intègrent une régularisation pour préserver la qualité du maillage; l’analyse précise leurs compromis: le LSM exploite des a priori, le DMM est indépendant des données, et le choix dépend des besoins.

Sur cette base, je développe DeepGeo, qui généralise le DMM à des géométries 3D complexes en apprenant automatiquement la déformation dans des espaces de haute dimension, simplifiant ainsi la paramétrisation de forme en optimisation. Il offre de larges libertés de déformation tout en imposant une régularité globale. Dans des études de cas--cercle 2D, aile NASA Common Research Model, et avion Blended-Wing-Body (BWB)--DeepGeo produit des formes optimisées dont les performances aérodynamiques sont comparables à celles de paramétrisations manuelles de pointe, tout en réduisant significativement l’effort manuel.

La thèse se tourne ensuite vers l’exploration de l’espace de conception, étape nécessaire avant l’optimisation pour le prototypage de concepts. Je propose DiffGeo, un modèle de diffusion en espace latent qui génère des formes aérodynamiques diverses et valides. En l’entraînant sur les codes latents issus du LSM, DiffGeo atteint une grande efficacité en données et garantit la validité des conceptions générées. DiffGeo montre aussi l’échantillonnage conditionnel: en guidant la diffusion par des objectifs géométriques complexes, on génère des formes répondant à des critères spécifiques. Enfin, j’ai participé au développement de Dflow-SUR, une diffusion guidée par la physique, efficace et fortement contrôlable. L’optimisation des bruits échantillonnés par rétropropagation des gradients d’un modèle substitut conduit à des profils et des ailes aux performances aérodynamiques améliorées.

Ces contributions forment un pipeline cohérent: de l’exploration efficiente de l’espace de conception à une optimisation hautement automatisée, rendue possible par des modèles de paramétrisation entièrement différentiables. Les résultats montrent que l’apprentissage profond géométrique traite des représentations de forme exigeantes et accélère le cycle de conception en automatisant une grande part du traitement géométrique et en offrant une génération rapide de concepts. Cette recherche interdisciplinaire ouvre la voie à des processus de conception plus exploratoires, plus efficaces et véritablement collaboratifs entre l’humain et l’IA.

\textbf{Mots-clés:} conception ingénierique par l’IA, apprentissage géométrique profond, optimisation aérodynamique des formes, paramétrisation de forme, conception générative

\end{otherlanguage}


%\endgroup			
%\vfill
