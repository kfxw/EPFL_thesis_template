\section{Related Work}
\label{ch3:sec:related_work}

In the CFD literature, the mesh representation has been widely applied in various tasks, to name a few recent works, including mesh deformation \cite{aa.Mi2022,aa.Stannard2022}, aerodynamic shape optimization \cite{aa.Li2022c,aa.Kaya2022}, multidisciplinary design \cite{aa.Wu2022,aa.Li2022,aa.Li2022}, etc. Typical examples arise in simulations with fluid-structure interactions, since the geometry and thus the associated mesh are changing over time as a result of the balance between the aerodynamic load and the mechanical structure. For such problems, difficulties to guarantee the mesh quality appear when large deformations occur~\cite{aa.Tian2014}. Similarly, in the contexts of aerodynamic optimization~\cite{aa.Lyu2015}, uncertainty quantification (UQ,~\cite{aa.Roy2018}) or sensitivity studies, numerous evaluations of various shapes and/or meshes are required, which again is difficult to automatize while guaranteeing a proper mesh quality for CFD. Specifically, both geometrical quality (aspect ratio, skewness, etc.) and physical requirements (boundary layers, shocks, etc.) have to be maintained when deforming the mesh, a property named here as Mesh Quality Preservation (MQP).

For these tasks, mesh representations are usually derived from explicit handcrafted parameterizations instead of directly from the mesh. Common options include fixed sampling schemes~\cite{aa.Poole2015}, nonuniform rational B-splines (NURBS)~\cite{aa.Toal2010}, control points for Free-Form Deformations (FFD)~\cite{aa.Sederberg1986, aa.Lamousin1994, aa.Kenway2010} or for Radial Basis Functions (RBF)~\cite{aa.deBoer2007}. Manual hyperparameter tuning is needed to adapt to different geometries, such as the number and the positions of control points, the values of supporting radius, etc. In all cases, the mesh ends up being described by a vector.

As for the linear dimension reduction methods, a popular way to produce a compact representation is to use the Proper Orthogonal Decomposition (POD) that mathematically derives a set of orthogonal modes. This can involve finding basis functions via Gram-Schmitt orthogonalization on a few geometries~\cite{aa.Robinson2001} or applying an SVD to create optimal orthogonal shape modes given a training dataset~\cite{aa.Poole2015,aa.Masters2017,aa.Li2019b,aa.Kedward2020}. However, reconstructing the geometry from a latent vector remains difficult and requires dedicated handcraft engineering for a specific task. The Active Subspace Model (ASM) \cite{aa.Constantine2014,aa.Li2019,aa.Lukaczyk2014,aa.Namura2017,aa.Grey2018} and Active Subspace Identification (ASI)\cite{aa.Bauerheim2016} reduce the dimension by analyzing the gradient of surrogate models. It is intended to limit the curse of dimensionality \cite{ai.Bellman1961} for surrogate-based optimization or uncertainty quantification. It uses random sampling methods to estimate  the real active subspace. Even though it requires handcrafted rules to generate valid samples, ASM's approximation error is upper bounded by the Poincaré constant that increases with dimensionality given a limited number of samples \cite{ai.Payne1960,ai.Beyer1999}. The Class/Shape function Transformation (CST)~\cite{aa.Kulfan2008} describes 2D airfoil shapes by the summation of Bernstein polynomial basis. Higher dimensional CST representation works for more complicated geometries but has a slower convergence rate when applied in the shape optimization task~\cite{aa.Ceze2009}. Non-linear approaches have also been proposed. For example, Generative Topographic Mapping (GTM)~\cite{aa.Viswanath2011} is used to  project a 30-dimensional design variable to a two dimensional latent vector. However, because GTM involves a Bayesian generative model, its latent representation cannot easily be integrated into a gradient-based pipeline. Furthermore the complexity of tuning hyperparameter for the radius basis grows exponentially with the latent space's dimensionality. Finally, all the models mentioned above are either linear projections or single nonlinear projections with Gaussian kernels.  By relying on a nonlinear deep neural network, our proposed model can learn more complex representations.

More recently, Generative Adversarial Network (GAN) has been used for novel geometry generation \cite{aa.Achour2020,aa.Du2020,aa.Chen2020}. They can be difficult to train and efforts have been made to stabilize their training and to filter out invalid results~\cite{aa.Li2020,aa.Li2021}. They are typically used for data sampling to generate novel shapes and augment training data for the subsequent parameterization. Hence, they serve as preconditioning modules in step-by-step frameworks and cannot contribute to gradient propagation in end-to-end pipelines. 

We took our inspiration from computer vision work that has convincingly demonstrated the ability of auto-encoders~\cite{ai.Dai2017,ai.Wu2018b}, variational auto-encoders~\cite{ai.Bagautdinov2018} and auto-decoders~\cite{ai.Tan1995,ai.Park2019c} to learn latent geometric representations and eliminate the explicit shape parameterization. The deep learning based models are able to learn accurate object surface representation automatically with the signed distance field. Recent progress makes the auto-decoder models fully differentiable \cite{ai.Remelli2020b}.
However, unlike in these approaches, our model is mesh-based and represents both the object surface and corresponding computational mesh. The Mesh Quality Preservation for CFD purposes is a major consideration, which was not part of computer vision research.
The deep learning algorithms for computer vision therefore have to be thoroughly adapted to the specificity of fluid simulations so that they become useful for the CFD community. Consequently, the main objective of our work is to propose a novel mesh representation and deformation framework, which (i) is handcrafted-parameter-free, (ii) encapsulates MQP, (iii) is fully differentiable for a direct implementation in gradient-based frameworks needed in optimization or uncertainty quantification studies, and additionally (iv) is flexible to the mesh type employed so that structured, unstructured or hybrid meshes can be used. This framework is developed here for 2D meshes and is validated on various CFD tasks, including the aerodynamic optimization of a 2D airfoil with several geometrical constraints.