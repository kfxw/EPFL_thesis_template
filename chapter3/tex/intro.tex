\section{Introduction}

Mesh manipulation is at the heart of Computational Fluid Dynamics (CFD) applications. However, generating appropriate computational meshes often involves much manual intervention that has to be repeated for each new shape under consideration. This is time-consuming for several reasons. First, the parameterization and its hyper-parameters depend heavily on the target object, for which strong geometric prior knowledge and a case-by-case analysis are always required. Second, the resulting model is specialized and rarely generalizes to different geometries or boundary conditions. Hence, re-meshing is required when the shape changes and much time is wasted in industrial practice. Finally, none of the existing approaches to automated mesh representation~\cite{aa.Robinson2001, aa.Poole2015, aa.Masters2017,aa.Li2019, aa.Kedward2020, aa.Viswanath2011, aa.Constantine2014,aa.Li2020,aa.Li2021} are designed to be differentiable so they can be integrated seamlessly into gradient-based frameworks, such as those that involve deep learning.

In this work, we propose a fully automated approach to redesigning a computational mesh as the target shape changes. It relies on a latent representation, i.e. a vector in a low-dimensional space, of both the object surface and the corresponding computational mesh. Through deep learning, the model learns geometric priors for the target objects while guaranteeing that the resulting meshes can be used to produce accurate simulations. To this end, it encodes an unstructured point cloud sampled from the object surface into a low-dimensional latent vector and then decodes it into an appropriately deformed CFD mesh. To preserve its quality, we introduce a regularization loss and a differentiable Active Model layer. The proposed model eliminates most manual steps, and avoids re-meshing without relying on any specific mesh parameterization. The method can also output various mesh types, such as structured, unstructrued or hybrid meshes. Additionally, the resulting computational meshes are fully differentiable with respect to the latent vector, which allows integration into deep learning pipelines or any gradient-based optimization frameworks. 

We develop and validate our approach on 2D airfoils. The paper is organized as follows.  In Sec.~\ref{ch3:sec:related_work}, we provide a literature review of related mesh representation methods. In Sec. \ref{ch3:sec:method}, we formulate and discuss the technical details of the proposed model. In Sec. \ref{ch3:sec:Experiments}, we conduct comprehensive experiments to prove the effectiveness of the proposed representation. The model is then integrated into an end-to-end shape optimization pipeline to demonstrate the benefits of full differentiation.
In Appendix, we explore the properties of learned latent space. Unelaborated mathematical and experimental details are also provided.