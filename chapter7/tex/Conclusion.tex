\section{Conclusions}
\label{ch7:sect:conculsion}
In this paper, we presented \textit{Dflow‑SUR}, a data-driven approach suitable for physics-guided generative inverse design that decouples flow matching inference from physical loss optimization by differentiating throughout the entire generative trajectory. Unlike traditional conditional training and energy-based inference approaches, which suffer from gradient collisions and asynchronous dynamics, \textit{Dflow‑SUR} evaluates the physical loss only at the terminal sample and back-propagates its gradient to the initial noise input. 

We validate the effectiveness of the proposed approach on both 2D airfoil and 3D wing design tasks. In the airfoil case, \textit{Dflow‑SUR} improves generation accuracy by up to four orders of magnitude compared to the best-performing traditional method, while reducing generation time by $74.47\%$. As a high-performance aerodynamic sampler in the 3D wing case, \textit{Dflow‑SUR} generates a more concentrated and elevated distribution under prescribed aerodynamic constraints, achieving an $11.8\%$ increase in mean $L/D$ over Latin Hypercube Sampling and a $6.5\%$ improvement over the energy-based approach. As a result, the $C_P$ distributions of the generated samples indicate that \textit{Dflow‑SUR} produces more aerodynamically reasonable configurations.

In conclusion, \textit{Dflow‑SUR} framework delivers three principal benefits for physics‑informed generative inverse design, as demonstrated through a series of comparative experiments. First, it offers \textbf{superior guidance controllability}; by separating the flow‑matching inference from physical‑loss optimization, \textit{Dflow‑SUR} eliminates the gradient collisions inherent to tightly coupled methods, enabling more thorough generative and physical‑optimization processes. Second, it provides \textbf{uncertainty control in early denoising}. In this context, \textit{Dflow‑SUR} evaluates only the final design $\mathbf{x}_1$ and back‑propagates its physical gradient to the initial noise $\mathbf{x}_0$ via the chain rule, thereby avoiding the surrogate-model uncertainties that arise from out-of-distribution intermediates. Third, \textbf{hyperparameter robustness}; unlike energy‑based approaches that require meticulous tuning of coefficients and cutoff times, \textit{Dflow‑SUR} operates with essentially no additional hyperparameters, facilitating straightforward batch deployment.

To recap, the primary objective of \textit{Dflow‑SUR} is to enable the incorporation of physical information into the generative process with higher accuracy and efficiency, rather than to replace traditional high-fidelity CFD-based design optimization paradigms. The generative model is intended to explore physically reasonable conceptual designs, serving as a front-end tool for design space exploration. Given the substantial computational cost associated with physics-based evaluation in computational mechanics—unlike typical AI tasks—\textit{Dflow‑SUR}'s decoupled framework, which separates physical loss from flow matching and restricts physics evaluation to plausible designs, offers a practical and scalable solution.