\section{Related Work}
\label{ch6:sec:related_work}

As a novel generative framework for DSE, \textit{DiffGeo} lies at the intersection of geometry modeling, deep generative learning and multi-disciplinary design optimization. To contextualize its contributions, we review related work across three areas: Design space exploration in aerospace, with a focus on geometry parameterization; deep generative models in ASO; and diffusion models.

\subsection{Design Space Exploration in Aerospace}
Design space exploration in aerospace has gone through several major stages, each marked by advances in geometry modeling and sampling methodology. The foundation was laid in the 1930s with the NACA airfoil series, where systematic variation of simple shape parameters, such as camber and thickness, enabled engineers to explore predefined families of airfoils~\cite{aa.Jacobs1933}. These early parametric studies provided smooth, physically interpretable shapes but constrained exploration to narrow subspaces of the design space. As computational methods matured, formal Design of Experiments (DoE)~\cite{aa.Fisher1935} techniques such as Latin Hypercube Sampling (LHS)~\cite{ai.McKay1979} were introduced to support broader and more uniform coverage of parameter spaces. CFD-based optimization studies, like \citet{aa.Hicks1978}, demonstrated the need for more flexible parameterizations that could balance geometric expressivity against numerical stability. The emergence of gradient-based optimization and adjoint sensitivity analysis then allowed high-dimensional shape manipulation, but continued to rely on analytic parameter bases like the Hicks-Henne bump functions or the Class/Shape Transformation (CST) method~\cite{aa.Kulfan2008}. While these parameterizations kept the design space tractable, they still imposed structural assumptions on shape variation. In response, surrogate-assisted DSE was introduced: Statistical models, such as Kriging~\cite{aa.Matheron1963}, response surfaces~\cite{aa.Box1951} and neural networks~\cite{ai.Rumelhart1986}, were built from structured DoE samples, enabling global search with significantly fewer high-fidelity evaluations~\cite{aa.Sacks1989, aa.Barthelemy1993}. These methods expanded exploration beyond local gradients but still required carefully crafted parameterizations and iterative optimization loops.

More recently, the field has shifted toward data-driven modeling of the geometry space for DSE. Latent space representations derived from principal component analysis~\cite{aa.Robinson2001,aa.Poole2015,aa.Masters2017,aa.Li2019,aa.Li2021b}, autoencoders~\cite{aa.DAgostino2018,aa.Rios2021,aa.Li2020}, variational autoencoders~\cite{aa.Yonekura2021,aa.Kou2023,aa.Swannet2024,aa.Wang2022} or autodecoders~\cite{aa.Wei2023, aa.Wei2023b} have been used to capture nonlinear shape variation across existing databases of feasible designs. These learned manifolds restrict sampling to valid and meaningful geometries, addressing the challenge that most random shape perturbations yield abnormal results.

While these latent‐space methods improve geometric expressivity, so far they have been paired primarily with rule-based static sampling or surrogate‐based search strategies. The exploration and exploitation of such DSE methods heavily rely on the design of the sampling strategy and the quality of the surrogate models. Deep generative models advance this line of work by learning a generative sampler, enabling the direct synthesis of novel designs within the learned space.

\subsection{Deep Generative Design for Aerodynamic Design}

Deep generative design methods have become increasingly popular in ASO and MDO, enabling rapid synthesis of novel geometries with minimal manual parameter tuning through data-driven training. Among these, \citet{aa.Chen2020} first applied GAN-based models~\cite{ai.Goodfellow2020} to 2D airfoil generation by embedding Bézier-curve parameterizations. Subsequent efforts extended these approaches using B-spline parameterizations~\cite{aa.Du2020} and mode decomposition~\cite{aa.Li2020,aa.Li2021}. Conditional GANs have also been used to accommodate constrained design objectives~\cite{aa.Achour2020,aa.Wang2022,aa.Lei2021}. While GANs can generate smooth and diverse samples, their adversarial training paradigm often suffers from instability, including mode collapse, and demands substantial data as well as architectural tuning~\cite{ai.Arjovsky2017}.

Invertible neural networks (INNs)~\cite{ai.Dinh2015,ai.Dinh2017} offer a complementary approach by learning bijective mappings between geometry and performance spaces. For example, \citet{aa.Glaws2022} demonstrated that INNs trained on airfoil–performance pairs can invert target performance metrics into multiple valid geometries in a single forward pass. However, such models require carefully curated paired datasets and strict balancing of forward and inverse loss functions, which introduces additional data engineering complexity.

More recently, diffusion-based models~\cite{ai.Ho2020} have emerged in aerodynamic design. Our previous work \textit{DiffAirfoil}~\cite{aa.Wei2024} was the first to introduce diffusion models into airfoil generation. It demonstrated the model's superiority over GANs under data-scarce conditions. Several follow-up studies explored conditional diffusion models for airfoil synthesis~\cite{aa.Graves2024,aa.Yonekura2024,aa.Wagenaar2024,aa.Yang2024}. However, these methods often adopt a Classifier-Free Guidance (CFG) mechanism that models the joint distribution of geometry and performance, which requires large-scale labeled datasets. This makes their training costs comparable to GANs or INNs. Furthermore, the lack of explicit geometric parameterization leads to instability in generation, often necessitating heavy post-processing or smoothing tricks to produce valid designs~\cite{aa.Graves2024,aa.Yonekura2024}.

Despite these advances, deep generative models still face challenges in training stability, data efficiency and controllability. Building on \textit{DiffAirfoil}, \textit{DiffGeo} aims to address these limitations. Rather than being as a monolithic model for end-to-end generative design, we position \textit{DiffGeo} as a tool to automate and enhance existing MDO workflows that produces high-quality design candidates for downstream analysis and optimization through rapid prototyping.

\subsection{Denoising Diffusion Models}

Modern denoising diffusion models~\cite{ai.SohlDickstein2015,ai.Ho2020,ai.Song2021c} have emerged as a powerful and stable generative framework over the past decade. These models operate by gradually corrupting data with Gaussian noise and then training a neural network to learn the reverse denoising process. Originally formulated using discrete-time Markov chains, recent advances reinterpret this process as solving stochastic differential equations (SDEs), which can be further reformulated into equivalent deterministic formulations such as probability flow ordinary differential equations (ODE)~\cite{ai.Song2021c,ai.Karras2022}, and continuous-time normalizing flows~\cite{ai.Lipman2022,ai.Liu2023f,ai.Albergo2023}.

Conditional diffusion models introduce guidance mechanisms to control the generation process toward desired outputs. \citet{ai.Dhariwal2021} proposed classifier guidance, which incorporates gradients from a pre-trained external classifier to bias diffusion sampling toward specific class labels, effectively trading off sample diversity for fidelity. As an alternative, \citet{ai.Ho2022} introduced Classifier-Free Guidance (CFG), which jointly trains conditional and unconditional score estimators and interpolates between them at sampling time. While CFG has become the foundation of recent large-scale diffusion models, it typically requires massive datasets and training budgets, making it impractical for data-scarce domains such as aerospace engineering. In contrast, \textit{DiffGeo} develops the energy-based guidance that decouples geometry generation from performance evaluation through differentiable energy functions, enabling task-specific guidance at sampling time while improving data efficiency and model reusability.

Diffusion methods have also been applied to 3D shape modeling, initially targeting denoising directly on point clouds~\cite{ai.Yang2019d,ai.Mao2023}. However, the data-driven nature of these approaches, combined with the lack of explicit shape parameterizations, often leads to irregular or invalid geometries—such as noisy point clusters, disconnected components, or self-intersecting surfaces. These issues are especially problematic in aerospace engineering, where high shape validity is critical and oscillatory or non-physical shapes must be avoided~\cite{aa.Masood2024}. Point cloud-based diffusion models struggle to guarantee watertight, smooth geometries suitable for downstream simulation and analysis. \textit{DiffGeo} addresses this limitation by introducing an automatic shape parameterization stage that learns a latent space constrained to valid geometry manifolds. The diffusion process is then learned within this latent space, which enables the generation of controlled 3D shapes that are inherently regular and simulation-ready, thus overcoming a key limitation of earlier diffusion and point cloud-based 3D modeling methods.