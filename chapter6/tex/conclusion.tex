\section{Conclusion}
\label{ch6:sec:conclusion}
This work presents \textit{DiffGeo}, a latent space diffusion-based generative framework for data-efficient and controllable design space exploration. By revisiting our prior conference work and reframing the generative process as a reusable shape sampler, we demonstrate that \textit{DiffGeo} can serve as a fundamental tool for real-world design workflows in aerospace and beyond. Through extensive experiments in both 2D airfoil and 3D turbomachinery blade cases, we validate three core capabilities of our approach: high data efficiency, task-agnostic adaptability, and controllability under high-dimensional constraints.

\textit{DiffGeo} achieves stable and diverse shape generation with only limited training data, with 50 samples in 2D and interpolated shapes from just six blade profiles in 3D. This drastically reduces the data barrier typically associated with deep generative models and makes it viable in realistic industrial contexts. Furthermore, we show that a single trained \textit{DiffGeo} model can be re-used across multiple design objectives by simply replacing the guidance energy functions, eliminating the need for retraining and decoupling geometric learning from task-specific tuning. This modular design opens new directions for combining generative modeling with physics solvers or expert-defined constraints. Finally, the ability to enforce complex geometric conditions, such as spanwise twist distributions or cross-sectional thickness requirements, proves that \textit{DiffGeo} is capable of handling real-world engineering needs that previous generative methods can hardly support.

From a broader perspective, \textit{DiffGeo} introduces a new paradigm for early-stage design exploration. Rather than relying on handcrafted parameterizations or trial-and-error sampling, designers can now directly generate feasible and diverse candidates that comply with desired performance constraints all in a one-shot and differentiable fashion. This greatly accelerates the conceptual design stage and enhances the quality of candidate designs available for downstream high-fidelity optimization. Our SBO experiments on 2D airfoils shows that this guided generative sampling strategy can be used to improve surrogate model's accuracy and final optimized performance, compared to baseline strategies based on random statistical sampling or historical data. Our 3D blade case study further confirms that \textit{DiffGeo} not only produces valid shapes outside the original linear interpolation space, but also leads to designs with higher aerodynamic efficiency.

While this work focuses on geometric constraints in aerodynamic shape design, the core framework of \textit{DiffGeo} is not limited to geometry alone. The latent diffusion sampler and energy-based conditioning mechanism form a general interface for embedding design principles into generative exploration. Although this paper does not yet incorporate physical design objectives, the structure of our method inherently accommodates extensions where guidance is informed by physics--such as aerodynamic performance, structural stress or thermal constraints--through differentiable approximations or adjoint solvers. More importantly, the theory is not constrained to single physical domain. In the future, \textit{DiffGeo} has the potential to support multidisciplinary design scenarios, where diverse engineering requirements can be jointly encoded and explored in a unified generative framework.

We believe this work paves the way for democratizing generative design. The ability to train models on small datasets, incorporate expert intuition as energy functions, and generate high-performing designs without intensive tuning makes \textit{DiffGeo} accessible to teams with limited data or ML infrastructure. This lowers the entry barrier to AI-driven design and enables non-specialist engineers to explore new ideas with greater freedom and confidence.

For future studies, we see several opportunities to extend this work. First, we plan to incorporate physics-informed guidance into the sampling process by integrating adjoint-based solvers, enabling the generative model to address performance objectives such as lift-to-drag ratio, stress distribution or thermal compliance. To make this feasible in practice, we also aim to develop more efficient conditional sampling algorithms that reduce the number of energy function evaluations, thereby limiting the computational cost and solver calls during sampling. Second, we intend to enhance the robustness of the framework, including the study of sampling stability when operating near the edges of the learned latent manifold of designs, and introducing uncertainty-aware strategies to quantify generation confidence or constraint satisfaction likelihood. Finally, we plan to embed \textit{DiffGeo} into an interactive design workflow, where human engineers collaborate with the generative model in the conceptual design stage to provide online constraints, interpret generated results and enable exploration in real time.

In summary, \textit{DiffGeo} establishes a new approach to data-driven design space exploration. Its demonstrated performance in both 2D and 3D cases shows that it can generate valid, novel and constraint-satisfying shapes under severe data limitations. We hope this work will inspire further research in controllable generative modeling for design, and contribute to a more flexible, automated and creative future for aerospace engineering.