\section{Related Work}
\label{ch5:sec:related_work}

DeepGeo spans across multiple research domains, including the shape parameterization algorithms in computer graphics and aerodynamic shape optimization studies, as well as self-prior models in deep learning research. Each part will be briefly reviewed, and the DeepGeo's technical prominence will be discussed afterward.

\subsection{Shape Parameterization in Aerodynamic Shape Optimization}
Shape parameterization in ASO refers to the mapping of explicit design variables to the object's surface mesh, a fundamental module for effectively solving ASO problems. \citet{aa.Jameson1988} introduced a direct conformal mapping from a circle to optimize airfoils, which provided the maximal design freedom in design space but lacked sufficient smoothness for CFD solvers~\cite{aa.Braibant1984}. Therefore, shape parameterization is necessary to reduce the design space dimensionality and introduce smoothness control. It can be broadly categorized as non-data-driven and data-driven approaches. 

\subsubsection{Non-Data-Driven Approaches}

Non-data-driven algorithms include both constructive and deformative methods~\cite{aa.Masters2017}. Constructive methods, such as polynomial/spline based methods~\cite{aa.Kulfan2008,aa.Braibant1984,aa.Farin1995} and partial differential equation methods~\cite{aa.Bloor1995,aa.Smith1995}, represent the airfoil surface based on specified parameters. Deformative methods deform an existing shape, including analytical methods~\cite{aa.Hicks1978,aa.Pickett1973,aa.Hager1992} and the free-form deformation (FFD)~\cite{aa.Sederberg1986, aa.Lamousin1994, aa.Kenway2010}. Early analytical approaches, like the Hicks-Henne bump function~\cite{aa.Hicks1978} and others~\cite{aa.Pickett1973,aa.Hager1992}, optimized the shape through linear combinations of basis functions. B-splines~\cite{aa.Braibant1984} and its variants, like the Bezier curve and nonuniform rational B-spline (NURBS)~\cite{aa.Farin1995}, use control points and piece-wise polynomials to define the geometry. Polynomial-based methods like parameterised sections (PARSEC)~\cite{aa.Sobieczky1999} and CST~\cite{aa.Kulfan2008} approximate surfaces with weighted sums of polynomials with different orders. While effective, these methods are suited for 2D curves. The FFD, originating from soft object animation in computer graphics~\cite{aa.Sederberg1986,ai.Barr1984,aa.Lamousin1994}, enables smooth continuous volume transformations based on control point movements. It is widely used in optimizing 3D objects like the CRM wing~\cite{aa.Lyu2015,aa.Wu2022} and the overall aircraft~\cite{aa.Secco2021}.

When applied to infinite-dimensional problems with finite-dimensional design variables, non-data-driven methods require compromising between the dimensionality for effective parameterization and the optimization complexity~\cite{aa.Ceze2009}. Additionally, the lack of self-adaptation leads to heavy reliance on expert knowledge and manual tuning for hyperparamters. As such, different settings can significantly impact optimization results. To address these limitations, DeepGeo bases on the neural network which provides universal approximation ability~\cite{ai.Barron1993} and mitigates the curse of dimensionality~\cite{ai.Barron1993,ai.Poggio2017}.

\subsubsection{Data-Driven Approaches}

Data-driven methods aim to reduce the dimension of design variables using existing geometry datasets, facilitating effective optimization for challenging problems or extracting patterns to reduce human intervention~\cite{aa.Li2022b}. Linear dimension reduction methods use proper orthogonal decomposition (POD) to derive a set of orthogonal modes. This can be done by Gram-Schmitt orthogonalization~\cite{aa.Robinson2001} or applying a singular value decomposition (SVD) to find orthogonal shape modes from a dataset~\cite{aa.Poole2015,aa.Li2019,aa.Kedward2020}. However, reconstructing the geometry from a latent vector remains challenging and requires dedicated manual designs for specific tasks. The active subspace model (ASM)~\cite{aa.Constantine2014,aa.Li2019b,aa.Lukaczyk2014,aa.Namura2017,aa.Grey2018} and active subspace identification (ASI)~\cite{aa.Bauerheim2016} reduce the dimensionality by analyzing the gradient from surrogate models for surrogate-based optimization or uncertainty quantification. Random sampling methods were used to estimate the real active subspace. However, generating valid samples requires handcrafted rules, and the approximation error is upper bounded by the Poincar\'{e} constant, which increases with dimensionality given a limited number of samples~\cite{ai.Payne1960,ai.Beyer1999}. Among the non-linear approaches, \citet{aa.Viswanath2011} proposed the generative topographic mapping (GTM) that projects a 30-dimensional design variable into a 2D latent vector. GTM's Bayesian generative model makes it challenging to integrate its latent representation into a gradient-based pipeline. More recently, generative models has been used for novel geometry generation \cite{aa.Achour2020,aa.Chen2020,aa.Li2020,aa.Li2021,aa.Wei2024,aa.Yang2024} to improve the quality of dimension reduction~\cite{aa.Li2020,aa.Li2021} or serve as a parameterization model directly~\cite{aa.Achour2020,aa.Chen2020}.

Data-driven methods typically demand large datasets for training~\cite{aa.Li2022b,aa.Wei2023,aa.Li2019}. DeepGeo stands out for its high data-efficiency as it requires only a single initial geometry for training, without requiring any additional user input comparing to the traditional ASO pipeline. This characteristic makes it a feasible solution for complex 3D geometries with limited data availability.

\subsection{Shape Parameterization in Computer Graphics}
Shape parameterization in computer graphics aims to create bijective mappings between two surfaces or volumes, with one domain represented as a mesh~\cite{ai.Sheffer2006}. This long-standing and active research field has diverse applications, like in mesh morphing, smoothing, remeshing, and so on. Creating one-to-one mappings between coordinate spaces inevitably introduces distortions that need to be minimized to retain desired mesh properties. Research has focused on surface mesh parameterization using cost functions that include distortion metrics, leading to the development of angle-preserving/conformal mappings, area-preserving/authalic mappings~\cite{ai.Floater2005,ai.Sheffer2006}, as-rigid-as-possible transformation~\cite{ai.Sumner2004,ai.Sorkine2007}, least squares conformal mapping~\cite{ai.Levy2002} and Dirichlet energy minimization methods~\cite{ai.Schreiner2004,ai.Smith2015,ai.Terzopoulos1987b}. DeepGeo extends the previous research focus to surface and triangulated meshes so that it works with complicated volumetric and unstructured CFD meshes. DeepGeo aligns more closely with the concept of a computer graphics parameterization model, as it establishes a mapping between volumetric CFD meshes instead of generating a design variable vector.

While early ASO shape parameterization methods were inspired by computer graphics, they are not directly applicable to volumetric CFD meshes.  DeepGeo’s key contribution lies in its regularization loss, which efficiently implements constrained volumetric mesh parameterization. Specifically, DeepGeo's regularization loss is derived from the continuous manifold learning theory, eliminating the need for mesh topology and allowing for flexible and highly sparse sampling for loss calculation, thus avoiding unaffordable time and memory consumption on large-scale CFD meshes. Then, DeepGeo uses a neural network, replacing finite difference and finite element approximations with analytical auto-differentiation. Additionally, it uses adjoint gradient descent optimization, which is more effective and computationally efficient than computer graphics parameterization models. Third, DeepGeo’s approach is independent of mesh topology, which avoids complex mesh processing, especially in cases of extreme cell irregularity where the simple triangulation/tetrahedralization assumption do not hold for CFD meshes.

\subsection{Deep Learning Model with Self-Prior}

Self-prior is an emerging research topic in the field of deep learning, where deep neural networks are trained to acquire domain-specific prior knowledge from a single data sample. The deep image prior~\cite{ai.Ulyanov2018} and its follow-ups have shown strong capabilities in multiple low-level vision tasks with self-supervision, such as image super-resolution, denoising, inpainting, dehazing and deblur. In 3D domain, the deep geometric prior~\cite{ai.Williams2019} introduced MLP models to reconstruct partial geometry of a point cloud, while Point2Mesh~\cite{ai.Hanocka2020} proposed to reconstruct the entire surface mesh from a point cloud.

DeepGeo uses a similar learning technique, but it is the first model that demonstrates how the self-prior can be effectively harnessed and manipulated with the mesh representation under the guidance of an external adjoint CFD solver.